\documentclass{beamer}
\usepackage[utf8]{inputenc}
\usepackage[T1]{fontenc}
\usepackage{fancyvrb}
\title{Deep Learning on the IMDB Dataset}
\date{WASP Deep Learning}
\author[Agents 47]{The Hitmen --- Agents 47}

\usetheme{wasp}

\graphicspath{{./graphics/}}

\begin{document}

\begin{frame}
  \titlepage
\end{frame}


\begin{frame}{Convnets}{Convolutional Neural Networks}

\begin{itemize}

\item[#Training] Using all data, we reached the best result after around 1 or 2
  epochs to around 93.5 \%. By using less, say 20 \% of the data, training
  improved for up to 4 epochs, but the validation results were considerably
  worse (73 \%). Note however that training with less data is considerably
  faster.
  
\item[#Batch] Smaller batches train slightly faster, but doesn't otherwise
  affect validation results much (Got 94.1 \% with batch size 64). Too big
  batches may however consume too much memory for efficient training.

\item[Reviews/Unique] Increasing these values can add a significant amount of
  data to the model, and thus \textbf{might} improve the result (93.7 \%), but
  will also take longer to train. Reducing it barely changs the result. (93.1
  \%).
 
\item[Dropout] 20\% dropout helps slightly: 94.7 \%.  40\% dropout helps a bit
  more: 95.5 \%. Even higher (80 \%) didn't improve the final result, but did
  make the learning process improve for all 10 epochs.

\item[Architecture] Krohn's architecture seems to perform slightly better, but
  takes a little longer to train than Chollet's. Comparing this with a network
  architecture with only dense layers (~88 \%), convolutional architectures seem
  superior.

\item[Smaller Network]
  
\end{itemize}
  
\end{frame}


\begin{frame}{RNN}{Recurrent Neural Networks}
  
\end{frame}

\begin{frame}{LSTM}{Long Short Term Memory}
  
\end{frame}


\begin{frame}
  \frametitle{Next Steps}
  From here, there are still a number of things one may want to do:
\end{frame}

\bgroup
\setbeamertemplate{background}{}
\setbeamercolor{background canvas}{bg=black}
% \setbeamertemplate{navigation symbols}{}
\begin{frame}[t,plain]{}{}
  \begin{center}
    {\tiny \textcolor{white}{The End}}
  \end{center}
\end{frame}
\egroup

\end{document}
