\documentclass{beamer}
\usepackage[utf8]{inputenc}
\usepackage[T1]{fontenc}
\usepackage{fancyvrb}
\title{Deep Learning for Computer Security}
\date{WASP Deep Learning}
\author[Agents 47]{Group 47}

\usepackage{svg}

\usepackage{graphicx}
\usepackage{caption}
\usepackage{subcaption}

\usepackage{pgfplots}
\pgfplotsset{compat=1.8}
\usepackage{pgfplotstable}
\usepgfplotslibrary{groupplots}

\usetheme{wasp}

\graphicspath{{./graphics/}}

\begin{document}

\begin{frame}
  \titlepage
\end{frame}

\begin{frame}{Machine Learning in Computer Security}

  \begin{itemize}

  \item Machine Learning is good at detecting patterns, and many computer
    viruses, worms, rootkits and so forth work by exploiting the computer in
    certain intrusion patterns.

  \item Thus, machine learning could feasibly be used as a part of a anti-virus
    suite to try to detect these, and thus consequently stop the offending
    process(es) or hacking attempts.

  \item Also, a lot of data already exist in data-bases for old viruses and
    vulnerabilities, so these could possibly be reused to train the networks.

  \end{itemize}

\end{frame}


\begin{frame}{Machine Learning for Security}{5-Pillars of Security}

  \note{Machine Learning is probably not very good for ensuring the 5-pillars of
    security:}

  \begin{description}

  \item[Confidentiality] Protection of information from disclosure to
    unauthorized individuals, systems, or entities.

  \item[Integrity] Protection of information, systems, and services from
    unauthorized modification or destruction.

  \item[Availability] Timely, reliable access to data and information services
    by authorized users.

  \item[Non-repudiation] The ability to correlate, with high certainty, a
    recorded action with its originating individual or entity.

  \item[Authentication] The ability to verify the identity of an individual or
    entity.

  \end{description}

\end{frame}

\begin{frame}{Machine Learning for Security}{3-Pillars for \textbf{Computer} Security}

  \note{In particular, in *computer* security, we usually only care about
    /confidientiality/, /integrity/, and /authentication/. However, one might
    assume that machine learning could be used for some aspect of this, however,
    machine learning is good for finding or creating /patterns/, and security in
    this context relies on the complete /absence/ of recognizable
    patterns. Thus, it is quite unlikely that deep learning will be used as a
    core feature of computer security.}

  \begin{itemize}
  \item Confidentiality
  \item Integrity
  \item Authentication
  \end{itemize}

\end{frame}


\begin{frame}{Machine Learning for Security}{Independent Validation}

  \begin{itemize}
  \item Independent validation of security algorithms
  \item Is the mathematics behind the algorithm sound?
  \item Does the algorithm perform the same for all kinds of inputs?
    \begin{itemize}
    \item Could the algorithm be trained in such a way that it would contain a
      backdoor?
    \end{itemize}
  \end{itemize}

  \note{Furthermore, the strength of many computer security algorithms depend on
    being rigorously analyzable. If someone develops a deep learning based
    algorithm for ensuring security, how could independent entities validate
    that it works as intended and does not contain some kind of backdoor for
    very specific inputs?}

  \note{We are already weary of accepting algorithms provided by the NSA. So
    would you trust a network they had trained?}

\end{frame}


\begin{frame}{Possible Problems with Machine Learning}

\begin{itemize}
\item Lack of data.
\item Company providing intrusion data $=$ Admitting they have been hacked.
\item GDPR + Hacked = Fines\ldots
\end{itemize}
    
\end{frame}

\bgroup
\setbeamertemplate{background}{}
\setbeamercolor{background canvas}{bg=black}
% \setbeamertemplate{navigation symbols}{}
\begin{frame}[t,plain]{}{}
  \begin{center}
    {\tiny \textcolor{white}{The End}}
  \end{center}
\end{frame}
\egroup

\end{document}
